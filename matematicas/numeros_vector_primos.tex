% Explicación inicial
Pensemos que cada número natural (excluido % línea 1
el 0) es un vector infinito de posiciones % línea 2
naturales (incluido el 0). En este caso, % línea 3
$V(N)[i]$ indica el exponente del $i$-ésimo % línea 4
primo en la factorización del número $N$. % línea 5
Llamemos $V(N)$ a la representación % línea 6
vectorial de $N$. % línea 7

% Multiplicación de vectores
Hacer $A \times B$ como números es % línea 8
sumar sus respectivos vectores. Es decir, % línea 9
$V(A \times B) = V(A) + V(B)$. % línea 10

% División de vectores
Análogamente, se tiene que % línea 11
$V\left(\frac{A}{B}\right) = V(A) - V(B)$. % línea 12

% Relación de divisibilidad
Lo interesante es notar que si un número % línea 13
divide a otro, entonces tiene un exponente % línea 14
menor o igual en cada factor primo. Es % línea 15
decir, % línea 16

$$A \mid B \iff V(A) \leq V(B)$$ % línea 17

(tomando $\leq$ posición a posición). % línea 18

% Máximo común divisor (GCD)
También se puede ver que: % línea 19

$$V(\text{GCD}(A,B)) = \min(V(A), V(B))$$ % línea 20

donde el mínimo se toma posición a % línea 21
posición, y % línea 22

% Mínimo común múltiplo (LCM)
$$V(\text{LCM}(A,B)) = \max(V(A), V(B))$$ % línea 23

donde el máximo se toma posición a % línea 24
posición. % línea 25

% Conclusión
Esto nos permite ver por qué GCD % línea 26
(Máximo Común Divisor) y LCM (Mínimo % línea 27
Común Múltiplo) tienen propiedades % línea 28
análogas a las de mínimo y máximo. % línea 29